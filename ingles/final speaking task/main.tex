\documentclass{article}
\usepackage[utf8]{inputenc}

\begin{document}

\section*{ESCENA 1: PROBLEMÁTICA}

KEVIN (Voz en off): La Universidad Tecnológica de Pereira (UTP) siempre ha sido un referente en el cuidado del medio ambiente. Sin embargo, como cualquier institución, enfrentaban un desafío importante en términos de energía.

LIA (Voz en off): Efectivamente, la UTP dependía en gran medida de la energía convencional, la cual tiene un alto impacto ambiental y no era sostenible a largo plazo.

KEVIN (Voz en off): Esto planteaba un problema significativo, ya que la universidad tenía una gran demanda energética para cubrir las necesidades de sus estudiantes, profesores y personal administrativo.

LIA (Voz en off): Además, la energía convencional generaba residuos peligrosos y contribuía al cambio climático. Era necesario encontrar una solución que permitiera a la UTP abastecerse de energía limpia y asequible.

KEVIN (Voz en off): Afortunadamente, se encontró una solución innovadora que cambiaría el panorama energético de la universidad.

\section*{ESCENA 2: SOLUCIÓN}

KEVIN: ¡Lia, estoy emocionado de contarte la solución que encontraron para la Universidad Tecnológica de Pereira!

LIA: Cuéntame, Kevin. Me encanta escuchar sobre iniciativas sostenibles.

KEVIN: Resulta que la UTP firmó un convenio con la Empresa de Energía de Pereira para instalar un sistema solar fotovoltaico en la universidad.

LIA: ¡Eso suena fantástico! ¿Cómo funciona exactamente?

KEVIN: El sistema solar fotovoltaico tiene una capacidad de 603kWp y estará distribuido en las áreas de cubierta de cuatro edificios de la UTP.

LIA: Entonces, ¿la universidad obtendrá toda su energía de fuentes solares?

KEVIN: Exacto. La energía eléctrica generada por el sistema fotovoltaico se conectará directamente a la red, sin necesidad de utilizar baterías. Esto permitirá que la universidad tenga una generación de energía 100\% limpia y sin producir residuos peligrosos.

LIA: Eso es realmente impresionante. La UTP se convertirá en la primera universidad de Colombia en implementar un sistema solar fotovoltaico de esta magnitud.

KEVIN: Así es, Lia. Esto demuestra el compromiso de la UTP con la protección del medio ambiente y el desarrollo sostenible. Además, la adopción de energía solar contribuirá a reducir la dependencia de la energía convencional y fomentará la investigación y el uso de tecnologías limpias.

LIA: Definitivamente es un gran paso hacia un futuro más sostenible. Ojalá más instituciones sigan este ejemplo.

KEVIN: Totalmente de acuerdo, Lia. La implementación de soluciones como esta nos acerca cada vez más al logro del SDG Clean and affordable energy y nos permite construir un mundo mejor para las generaciones futuras.

\textbf{FIN DE LA ESCENA}

\section*{SCENE 1: ISSUE}

\textbf{KEVIN (Voiceover):} The Universidad Tecnológica de Pereira (UTP) has always been a reference in environmental care. However, like any institution, they faced a significant challenge in terms of energy.

\textbf{LIA (Voiceover):} Indeed, UTP was heavily dependent on conventional energy, which had a high environmental impact and was not sustainable in the long term.

\textbf{KEVIN (Voiceover):} This posed a significant problem, as the university had a high energy demand to meet the needs of its students, teachers, and administrative staff.

\textbf{LIA (Voiceover):} Additionally, conventional energy generated hazardous waste and contributed to climate change. It was necessary to find a solution that would allow UTP to supply itself with clean and affordable energy.

\textbf{KEVIN (Voiceover):} Fortunately, an innovative solution was found that would change the university's energy landscape.

\section*{SCENE 2: SOLUTION}

\textbf{KEVIN:} Lia, I'm excited to tell you about the solution they found for the Universidad Tecnológica de Pereira!

\textbf{LIA:} Tell me, Kevin. I love hearing about sustainable initiatives.

\textbf{KEVIN:} It turns out that UTP signed an agreement with the Empresa de Energía de Pereira to install a solar photovoltaic system at the university.

\textbf{LIA:} That sounds fantastic! How does it work exactly?

\textbf{KEVIN:} The solar photovoltaic system has a capacity of 603kWp and will be installed on the rooftops of four buildings at UTP.

\textbf{LIA:} So, will the university get all its energy from solar sources?

\textbf{KEVIN:} Exactly. The electrical energy generated by the photovoltaic system will be directly connected to the grid, without the need for batteries. This will allow the university to have 100% clean energy generation without producing hazardous waste.

\textbf{LIA:} That's truly impressive. UTP will become the first university in Colombia to implement a solar photovoltaic system of this magnitude.

\textbf{KEVIN:} That's right, Lia. This demonstrates UTP's commitment to environmental protection and sustainable development. Moreover, the adoption of solar energy will help reduce dependence on conventional energy and promote research and the use of clean technologies.

\textbf{LIA:} It's definitely a big step towards a more sustainable future. I hope more institutions follow this example.

\textbf{KEVIN:} I completely agree, Lia. The implementation of solutions like this brings us closer to achieving the SDG Clean and affordable energy and allows us to build a better world for future generations.

\textbf{END OF SCENE}

\end{document}
