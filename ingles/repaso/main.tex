\documentclass{article}

% Paquetes necesarios
\usepackage[utf8]{inputenc} % Codificación de caracteres
\usepackage[spanish]{babel} % Idioma principal
\usepackage[left=1cm, right=1cm, top=1cm, bottom=2cm]{geometry}



% Título y autor del documento
\title{Título del Documento}
\author{Tú Nombre}
\date{\today}

\begin{document}

% \maketitle

% Contenido del documento
\section*{Topic: Use of Conjunctions}
Conjunctions are important parts of speech that connect words, phrases, or sentences within a sentence. They help to establish relationships and coherence between different elements of a sentence. Here is a general summary of the use of conjunctions:

\begin{enumerate}
    \item Coordinating Conjunctions:
    Coordinating conjunctions join words, phrases, or independent clauses of equal importance. The most common coordinating conjunctions are: "and," "but," "or," "nor," "for," "so," and "yet."
    Example:
    \begin{itemize}
        \item She likes to read books and write poems.
        \item He wanted to go for a walk, but it started raining.
        \item You can choose either the red shirt or the blue one.
    \end{itemize}
    \item Subordinating Conjunctions:
    Subordinating conjunctions introduce dependent clauses that rely on an independent clause to form a complete sentence. These conjunctions show a relationship of time, cause and effect, contrast, or condition. Common subordinating conjunctions include: "after," "although," "because," "if," "since," "until," "when," etc.
    Example:
    \begin{itemize}
        \item After she finished her homework, she went to bed.
        \item Although it was raining, they decided to go for a walk.
        \item Because he studied hard, he passed the exam.
    \end{itemize}
    \item Correlative Conjunctions:
    Correlative conjunctions are used in pairs to connect sentence elements that have equal importance. Some common correlative conjunctions are: "either...or," "neither...nor," "both...and," "not only...but also," "whether...or."
    Example:
    \begin{itemize}
        \item Either you come with us, or you stay here.
        \item She neither likes coffee nor tea.
        \item Both the dog and the cat are sleeping.
    \end{itemize}
\end{enumerate}

\section*{Topic: Phrases vs. Sentences}
Phrases and sentences are two fundamental components of language. Here's a general distinction between them:
\begin{enumerate}
    \item Phrases:
    A phrase is a group of words that functions together to convey a specific meaning, but it does not express a complete thought or contain a subject and a predicate. Phrases can act as nouns, adjectives, adverbs, or other parts of speech, and they are often used to provide additional information within a sentence.
    
    Example:
    \begin{itemize}
        \item The tall man with a hat
        \item Running in the park
        \item After the rain stopped
    \end{itemize}
    \item Sentences:
    A sentence is a group of words that expresses a complete thought, contains a subject and a predicate, and forms a grammatically independent unit. A sentence can be a statement, a question, a command, or an exclamation.
    
    Example:
    \begin{itemize}
        \item She is reading a book.
        \item Are you going to the party?
        \item Please close the door.
        \item What a beautiful sunset!
    \end{itemize}
\end{enumerate}

Remember that while phrases are components of sentences, they do not function independently as complete thoughts. Sentences, on the other hand, are self-contained units of communication.

\section*{Punctuation rules.}
Punctuation rules are essential for conveying meaning, clarity, and organization in written language. Here is a general overview of some common punctuation rules:
\begin{enumerate}
    \item Period (.) :

    Use a period at the end of a declarative sentence or an imperative sentence that is not a direct command or request.
    Example: She went to the store. Please sit down.
    \item Question Mark (?) :

    Use a question mark at the end of a direct question.
    Example: Where are you going?
    \item Exclamation Mark (!) :

    Use an exclamation mark at the end of an exclamatory sentence or to express strong emotion or emphasis.
    Example: What a beautiful day! Stop!
    \item Comma (,) :
    \begin{itemize}
        \item Use a comma to separate items in a list.
        Example: I need eggs, milk, and bread.
        
        \item Use a comma to separate independent clauses in a compound sentence.
        Example: She walked to the park, and he rode his bike.
        
        \item Use a comma to set off introductory elements or nonessential information within a sentence.
        Example: However, I still want to go. The book, which is red, belongs to Sarah.
    \end{itemize}
    \item Semicolon (;) :

    Use a semicolon to separate two closely related independent clauses without a coordinating conjunction.
    Example: I have a meeting at 9 a.m.; I need to prepare for it.
    \item Colon (:) :

    Use a colon to introduce a list, explanation, or example.
    Example: Please bring the following items: a pen, paper, and a calculator.
    \item Quotation Marks (" ") :

    Use quotation marks to indicate direct speech or a quotation.
    Example: She said, "I'll be there soon."
    \item Apostrophe (') :

    Use an apostrophe to indicate possession or contraction.
    Example: John's book (possession). It's raining (contraction of "it is").
    \item Dash (—) :

    Use a dash to indicate a sudden break or interruption in thought.
    Example: The weather—sunny and warm—was perfect for a picnic.
\end{enumerate}
These are just a few punctuation rules, and there are more specific guidelines for different punctuation marks. Understanding and following these rules can help enhance clarity, coherence, and readability in your writing.

\section*{Modal verbs and expressions for recommendation, obligation, prohibition, necessity, advice.}
Modal verbs and expressions are used to express various functions, such as recommendation, obligation, prohibition, necessity, and advice. Here are some examples of modal verbs and expressions for each of these functions:

\begin{enumerate}
    \item Recommendation:
    \begin{itemize}
        \item Modal verb: should
        Example: You should try the new restaurant in town. It has great reviews.
        \item Expression: I recommend (that)
        Example: I recommend that you read this book. It's very informative.
    \end{itemize}
    \item Obligation:
    \begin{itemize}
        \item Modal verb: must
        Example: Students must submit their assignments by the deadline.
        \item Expression: have to
        Example: I have to attend a meeting this afternoon.
    \end{itemize}
    \item Prohibition:
    \begin{itemize}
        \item Modal verb: must not / mustn't
        Example: You must not enter this area without proper authorization.
        \item Expression: It is forbidden to / You are not allowed to
        Example: It is forbidden to smoke in this building. You are not allowed to use your cell phone during the exam.
    \end{itemize}
    \item Necessity:
    \begin{itemize}
        \item Modal verb: need to
        Example: I need to finish this project by tomorrow.
        \item Expression: It is necessary to
        Example: It is necessary to wear a seatbelt while driving.
    \end{itemize}
    \item Advice:
    \begin{itemize}
        \item Modal verb: could / should
        Example: You could try taking a different route to avoid traffic. You should get more rest.
        \item Expression: It would be advisable to
        Example: It would be advisable to save some money for emergencies.
    \end{itemize}
\end{enumerate}
Remember that the choice of modal verb or expression depends on the specific context and the degree of recommendation, obligation, prohibition, necessity, or advice you want to convey.

\section*{Using conditionals (zero, first, second, and third conditional).}
Conditionals are used to express hypothetical or conditional situations and their potential outcomes. There are different types of conditionals: zero, first, second, and third conditional. Here are examples of each type:

\begin{enumerate}
    \item Zero Conditional:
    \begin{itemize}
        \item Form: If + present simple, present simple
        \item Function: Describing general truths or facts.
    \end{itemize}
    Example: If you heat water to 100 degrees Celsius, it boils.
    Explanation: This conditional expresses a cause-and-effect relationship that is always true. Whenever water is heated to 100 degrees Celsius, it boils.
    \item First Conditional:
    \begin{itemize}
        \item Form: If + present simple, will + base form verb
        \item Function: Expressing real or possible future situations.
    \end{itemize}
    Example: If it rains tomorrow, I will stay at home.
    Explanation: This conditional describes a possible future outcome. If it actually rains tomorrow, the speaker will choose to stay at home.
    \item Second Conditional:
    \begin{itemize}
        \item Form: If + past simple, would + base form verb
        \item Function: Expressing hypothetical or unreal situations in the present or future.
    \end{itemize}
    Example: If I won the lottery, I would travel around the world.
    Explanation: This conditional presents an unreal or unlikely situation. The speaker is imagining winning the lottery and the consequent action of traveling.
    \item Third Conditional:
    \begin{itemize}
        \item Form: If + past perfect, would have + past participle
        \item Function: Referring to hypothetical or unreal situations in the past.
    \end{itemize}
    Example: If I had studied harder, I would have passed the exam.
    Explanation: This conditional talks about a situation that didn't happen in the past. The speaker didn't study hard enough, and as a result, they didn't pass the exam.
\end{enumerate}
It's important to note that these examples provide a basic understanding of the different conditional types. There can be variations and additional nuances in their usage, depending on the specific context.

\section*{SDG 6: Clean water and sanitation.}
SDG 6, which stands for Sustainable Development Goal 6, is focused on ensuring access to clean water and sanitation for all. Here is a general summary of SDG 6:

SDG 6: Clean Water and Sanitation

Goal: Ensure availability and sustainable management of water and sanitation for all.

Summary:
SDG 6 aims to address the global challenges related to water scarcity, water pollution, inadequate sanitation, and lack of access to clean drinking water and basic sanitation facilities. It recognizes the importance of clean water and sanitation as fundamental human rights and crucial factors for human health, well-being, and sustainable development.

Key Aspects and Targets:
\begin{enumerate}
    \item Access to Water and Sanitation: Ensure universal and equitable access to safe and affordable drinking water, sanitation, and hygiene facilities.
    \begin{itemize}
        \item Target 6.1: Achieve universal and equitable access to safe and affordable drinking water for all.
        \item Target 6.2: Achieve access to adequate and equitable sanitation and hygiene for all, and end open defecation.
    \end{itemize}
    \item Water Quality and Efficiency: Improve water quality, reduce pollution, and increase water-use efficiency across all sectors.
    \begin{itemize}
        \item Target 6.3: Improve water quality by reducing pollution, eliminating dumping, and minimizing the release of hazardous chemicals.
        \item Target 6.4: Increase water-use efficiency and ensure sustainable withdrawals of freshwater resources.
    \end{itemize}
    \item Integrated Water Resources Management: Implement integrated water resources management at all levels to ensure sustainable use of water resources.
    \begin{itemize}
        \item Target 6.5: Implement integrated water resources management, including appropriate cooperation and coordination among stakeholders.
    \end{itemize}
    \item Water-related Ecosystems: Protect and restore water-related ecosystems, including forests, mountains, wetlands, rivers, and lakes.
    \begin{itemize}
        \item Target 6.6: Protect and restore water-related ecosystems, including forests, mountains, wetlands, rivers, and lakes.
    \end{itemize}
    \item International Cooperation and Capacity Building: Strengthen international cooperation and support capacity-building initiatives to enhance water and sanitation management.
    \begin{itemize}
        \item Target 6.a: Support developing countries in water and sanitation-related activities and programs.
        \item Target 6.b: Promote the participation of local communities in water and sanitation management.
    \end{itemize}
\end{enumerate}
By focusing on SDG 6, countries and stakeholders work towards ensuring clean water and sanitation for all, promoting sustainable water management practices, and improving the overall well-being and living conditions of people worldwide.

Note: This is a general summary of SDG 6. The specific targets and indicators outlined by the United Nations may provide further details and metrics for measuring progress towards achieving this goal.

\section*{SDG 7: Clean and affordable energy.}
SDG 7, which stands for Sustainable Development Goal 7, focuses on ensuring access to clean and affordable energy for all. Here is a general summary of SDG 7:

SDG 7: Clean and Affordable Energy

Goal: Ensure access to affordable, reliable, sustainable, and modern energy for all.

Summary:
SDG 7 aims to address the global challenges related to energy access, energy poverty, and environmental sustainability. It recognizes the importance of clean and affordable energy as a catalyst for economic development, poverty eradication, and environmental protection.

Key Aspects and Targets:
\begin{enumerate}
    \item Universal Energy Access: Ensure universal access to affordable, reliable, and modern energy services.
    \begin{itemize}
        \item Target 7.1: Ensure universal access to affordable and reliable energy services.
        \item Target 7.2: Increase the share of renewable energy in the global energy mix.
    \end{itemize}
    \item Energy Efficiency: Enhance energy efficiency in all sectors to promote sustainable energy use.
    \begin{itemize}
        \item Target 7.3: Double the global rate of improvement in energy efficiency.
        \item Target 7.4: Enhance international cooperation to facilitate access to clean energy research and technologies.
    \end{itemize}
    \item Renewable Energy: Increase the share of renewable energy in the global energy mix.
    \begin{itemize}
        \item Target 7.5: By 2030, achieve a substantial increase in the share of renewable energy in the global energy mix.
        \item Target 7.a: Enhance international cooperation to facilitate access to clean energy research and technologies.
    \end{itemize}
    \item Infrastructure and Technology: Enhance infrastructure and technology for clean and sustainable energy.
    \begin{itemize}
        \item Target 7.b: Expand and upgrade energy services and infrastructure, particularly in developing countries.
    \end{itemize}
    \item Energy Access for Least Developed Countries: Support the needs of least developed countries in accessing clean and affordable energy.
    \begin{itemize}
        \item Target 7.c: Increase international support for clean energy development in developing countries, particularly least developed countries.
    \end{itemize}
\end{enumerate}
By focusing on SDG 7, countries and stakeholders work towards ensuring access to clean and affordable energy sources, promoting energy efficiency and renewable energy technologies, and advancing sustainable energy practices to mitigate climate change and foster sustainable development.

Note: This is a general summary of SDG 7. The specific targets and indicators outlined by the United Nations may provide further details and metrics for measuring progress towards achieving this goal.

\section*{SDG 13: Climate action, SDG 14: Life below water, SDG 15: Life on land.}
SDG 13: Climate Action, SDG 14: Life Below Water, and SDG 15: Life on Land are three interconnected Sustainable Development Goals that address environmental sustainability and conservation. Here is a mix of these topics:

\begin{enumerate}
    \item Climate Action (SDG 13):
    SDG 13 focuses on urgent action to combat climate change and its impacts. It aims to strengthen resilience and adaptive capacity, promote sustainable practices, and mobilize resources to address climate-related challenges.
    \item Life Below Water (SDG 14):
    SDG 14 aims to conserve and sustainably use marine and coastal ecosystems. It addresses issues such as marine pollution, overfishing, ocean acidification, and the conservation of coastal and marine resources to ensure the sustainable livelihoods of communities dependent on the oceans.
    \item Life on Land (SDG 15):
    SDG 15 focuses on protecting, restoring, and promoting the sustainable use of terrestrial ecosystems. It addresses issues such as deforestation, biodiversity loss, land degradation, and the conservation of forests, wildlife, and natural habitats.
\end{enumerate}

Interconnections and Common Themes:

\begin{itemize}
    \item Climate change affects both marine and terrestrial ecosystems, including their biodiversity, habitats, and natural resources.
    \item The health of oceans and land ecosystems is crucial for mitigating climate change and adapting to its impacts.
    \item Sustainable land management practices can help sequester carbon, reducing greenhouse gas emissions and combating climate change.
    \item Sustainable management and conservation of marine and terrestrial ecosystems contribute to the overall well-being of communities, supporting livelihoods, food security, and biodiversity preservation.
\end{itemize}

Overall, the three SDGs emphasize the need for collective action, sustainable practices, and integrated approaches to address climate change, protect marine and terrestrial environments, and promote the interconnectedness of life below water and life on land.
\end{document}
