\documentclass[]{article}
\usepackage[utf8]{inputenc}
\usepackage[spanish]{babel}

\begin{document}
Línea 7: Se incluyen las librerías necesarias para el programa.\\
Línea 10: Se define la estructura Data que representa una palabra y su contador.\\
Línea 14: Se define la estructura Nodo que representa un nodo en el árbol binario de búsqueda.\\
Línea 18: Se declaran los prototipos de las funciones leerTeclado() y leerArchivo().\\
Línea 20: La función principal main() comienza.\\
Línea 22: Se declara la variable option para almacenar la opción elegida por el usuario.\\
Línea 24: Se inicia un bucle while que se repetirá hasta que la opción sea 3 (salir).\\
Línea 27: Se muestra el menú al usuario utilizando la función cout.\\
Línea 33: Se lee la opción ingresada por el usuario utilizando la función cin.\\
Línea 35: Se utiliza una estructura de control switch para ejecutar el código correspondiente a la opción seleccionada.\\
Línea 39: Si la opción es 1, se llama a la función leerArchivo().\\
Línea 41: Si la opción es 2, se llama a la función leerTeclado().\\
Línea 43: Si la opción es 3, se muestra un mensaje de despedida y se sale del bucle.\\
Línea 45: Si la opción no coincide con ninguna de las anteriores, se muestra un mensaje de opción inválida.\\
Línea 48: Se cierra el bucle while.\\
Línea 50: Se retorna 0 para indicar que la ejecución del programa ha sido exitosa.\\
Línea 54: La función agregar\_palabra() comienza.\\
Línea 56: Si el puntero raiz es nulo, se crea un nuevo nodo y se asigna la palabra y el contador.\\
Línea 59: Si la palabra es igual a la palabra del nodo actual, se incrementa el contador.\\
Línea 62: Si la palabra es menor que la palabra del nodo actual, se llama recursivamente a la función con el subárbol izquierdo.\\
Línea 64: Si la palabra es mayor que la palabra del nodo actual, se llama recursivamente a la función con el subárbol derecho.\\
Línea 68: La función contar\_palabras\_diferentes() comienza.\\
Línea 70: Si el nodo raíz es nulo, se retorna 0.\\
Línea 72: Se retorna la suma de 1 más el resultado de llamar recursivamente a la función con el subárbol izquierdo y el subárbol derecho.\\
Línea 76: La función imprimir\_arbol() comienza.\\
Línea 78: Si el nodo raíz no es nulo, se realiza un recorrido en orden del árbol.\\
Línea 79: Se muestra la palabra y su contador asociado en el nodo actual.\\
Línea 80: Se llama recursivamente a la función con el subárbol izquierdo.\\
Línea 81: Se llama recursivamente a la función con el subárbol derecho.\\
Línea 85: La función liberar\_arbol() comienza.\\
Línea 87: Si el nodo raíz no es nulo, se realiza un recorrido postorden del árbol.\\
Línea 88: Se llama recursivamente a la función con el subárbol izquierdo.\\
Línea 89: Se llama recursivamente a la función con el subárbol derecho.\\
Línea 90: Se libera la memoria asignada para el nodo actual.\\
Línea 95: La función leerTeclado() comienza.\\
Línea 97: Se declara el puntero arbol para representar la raíz del árbol binario de búsqueda.\\
Línea 98: Se declara el arreglo palabra para almacenar las palabras ingresadas por el usuario.\\
Línea 99: Se declaran variables para almacenar el total de palabras y el total de palabras diferentes.\\
Línea 101: Se inicia un bucle while que se repetirá hasta que se ingrese el carácter '!'.\\
Línea 102: Se muestra un mensaje para que el usuario ingrese una palabra.\\
Línea 103: Se lee la palabra ingresada por el usuario.\\
Línea 105: Si la palabra es igual a '!', se muestra un mensaje de finalización y se realizan las acciones correspondientes.\\
Línea 109: Se incrementa el contador de total de palabras.\\
Línea 112: Se convierte la palabra a minúsculas utilizando la función tolower().\\
Línea 113: Se llama a la función agregar\_palabra() para agregar la palabra al árbol.\\
Línea 118: La función leerArchivo() comienza.\\
Línea 120: Se declara el puntero arbol para representar la raíz del árbol binario de búsqueda.\\
Línea 121: Se declara el arreglo palabra para almacenar las palabras leídas desde el archivo.\\
Línea 122: Se declaran variables para almacenar el total de palabras y el total de palabras diferentes.\\
Línea 124: Se abre el archivo "data.txt" en modo lectura.\\
Línea 127: Si el archivo es nulo, se muestra un mensaje de error y se retorna 1.\\
Línea 130: Se inicia un bucle while que se repetirá hasta que se alcance el final del archivo.\\
Línea 132: Se incrementa el contador de total de palabras.\\
Línea 135: Se convierte la palabra a minúsculas utilizando la función tolower().\\
Línea 136: Se llama a la función agregar\_palabra() para agregar la palabra al árbol.\\
Línea 141: Se cierra el archivo.\\
Línea 143: Se muestra la frecuencia de las palabras en el árbol utilizando la función imprimir\_arbol().\\
Línea 144: Se calcula el total de palabras diferentes utilizando la función contar\_palabras\_diferentes().\\
Línea 146: Se muestra el total de palabras y el total de palabras diferentes.\\
Línea 147: Se libera la memoria asignada para el árbol utilizando la función liberar\_arbol().\\
Línea 151: Se retorna 0 para indicar que la ejecución de la función ha sido exitosa.\\
\end{document}