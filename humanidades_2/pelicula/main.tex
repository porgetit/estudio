\documentclass[12pt]{article}

% Paquetes necesarios
\usepackage[spanish]{babel}
\usepackage[utf8]{inputenc}
\usepackage[T1]{fontenc}
\usepackage{times}
\usepackage{graphicx}
\usepackage{fancyhdr}
\usepackage{setspace}

% Establecer los márgenes
\usepackage[left=2.5cm, right=2.5cm, top=3cm, bottom=3cm]{geometry}

% Configuración de encabezado y pie de página
\pagestyle{fancy}
\fancyhf{}
\rhead{Kevin Esguerra Cardona}
\lhead{Un futuro prometedor}
\rfoot{\thepage}

% Título, autor y fecha
\title{Un futuro prometedor}
\author{Kevin Esguerra Cardona}
\date{\today}

\begin{document}

% Portada
\maketitle

La película "Transcendence" presenta una visión fascinante del futuro de la humanidad y la tecnología, y plantea la posibilidad de que algún día podamos transferir nuestra conciencia a un cuerpo mecánico o digital. Aunque algunas personas pueden sentir cierto grado de aprensión al pensar en este tipo de transición, creo que hay buenas razones para pensar que esta transformación podría ser beneficiosa para la humanidad, y que las actuales generaciones de estudiantes deberían ser parte activa de ella.

En primer lugar, la evolución tecnológica ha estado ocurriendo de manera constante desde los años 60'. La capacidad de procesamiento de las computadoras se ha duplicado cada 18 meses, y la tecnología de inteligencia artificial y robótica ha avanzado significativamente. Hoy en día, la tecnología está en todas partes, y ha cambiado radicalmente la forma en que vivimos, trabajamos y nos relacionamos con los demás. Por lo tanto, no sería irracional pensar que la siguiente etapa de la evolución tecnológica será la transición de la conciencia humana a un cuerpo mecánico o digital.

En segundo lugar, hay muchas ventajas potenciales de esta transición. En primer lugar, podría significar una extensión significativa de la vida humana. Los cuerpos mecánicos o digitales pueden ser más resistentes y menos propensos a enfermedades y lesiones, lo que podría permitir a las personas vivir más tiempo y con una mejor calidad de vida. Además, la transición podría permitir a las personas superar las limitaciones físicas y mentales, lo que podría mejorar la calidad de vida y aumentar la capacidad de las personas para contribuir a la sociedad.

En tercer lugar, las actuales generaciones de estudiantes deberían estar en el centro de esta transición. Estos jóvenes son los que tienen la mayor comprensión y experiencia en el uso de la tecnología. Han crecido en un mundo completamente diferente al de sus padres, donde la tecnología ha sido una parte integral de sus vidas desde el principio. Por lo tanto, están en la mejor posición para liderar la transición hacia una forma de vida más tecnológica y avanzada.

En conclusión, creo que la transición de la conciencia humana a un cuerpo mecánico o digital es una posibilidad emocionante que podría ofrecer grandes beneficios a la humanidad. Las actuales generaciones de estudiantes son las que deben liderar este cambio, utilizando su comprensión y experiencia en el uso de la tecnología para mejorar la vida humana. A medida que la tecnología sigue avanzando, es importante que consideremos estas posibilidades y trabajemos juntos para asegurarnos de que el futuro de la humanidad sea brillante y prometedor.

\end{document}
