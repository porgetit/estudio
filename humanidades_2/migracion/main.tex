\documentclass{article}
\usepackage[utf8]{inputenc}

\title{Migración.}
\author{Kevin Esguerra Cardona}
\date{\today}

\begin{document}
\maketitle

\section*{Reflexión.}

Desde el principio de los tiempos, la migración ha sido un elemento fundamental de la historia humana. Desde las tribus nómadas hasta las grandes olas migratorias de la actualidad, los seres humanos siempre han buscado un mejor lugar para vivir y prosperar.

En este sentido, es fascinante pensar en la posibilidad de que nuestras raíces migratorias puedan estar ligadas a grandes personajes de la historia. Quizás nuestros antepasados emigraron siguiendo las huellas de Alejandro Magno o de Genghis Khan, o tal vez fueron parte de las grandes migraciones europeas que dieron forma a los Estados Unidos.

La idea de que nuestras vidas están conectadas con figuras históricas importantes es emocionante, y nos permite sentir que somos parte de algo más grande que nosotros mismos. Pero al mismo tiempo, es importante recordar que cada uno de nosotros es un individuo único, con nuestra propia historia y nuestro propio camino en la vida.

Al reflexionar sobre nuestro pasado migratorio, también es importante recordar las dificultades y los sacrificios que nuestros antepasados tuvieron que enfrentar. Desde los largos viajes en barco hasta las duras condiciones de vida en los nuevos lugares, la migración siempre ha sido una experiencia desafiante.

En la actualidad, muchos de nosotros seguimos emigrando en busca de nuevas oportunidades y mejores condiciones de vida. Pero debemos recordar que los desafíos que enfrentamos hoy son similares a los que enfrentaron nuestros antepasados en el pasado.

En última instancia, nuestra historia migratoria nos conecta con un pasado más amplio y nos recuerda la importancia de la empatía y la compasión hacia aquellos que están emigrando en la actualidad. Al recordar nuestros propios antecedentes migratorios, podemos entender mejor las dificultades que enfrentan los demás y trabajar juntos para construir un mundo más justo y equitativo para todos.

\newpage
\section*{Palabras.}
\subsection*{Raíz indígena:}
    \begin{enumerate}
        \item Tomate - Fruto comestible originario de América, utilizado en la gastronomía de todo el mundo. Raíz lingüística: náhuatl (tomatl).
        \item Quetzal - Ave sagrada en la cultura mesoamericana, cuya pluma se usaba para la decoración y la vestimenta de los líderes. Raíz lingüística: náhuatl (quetzalli).
        \item Cacao - Fruto utilizado para hacer chocolate, originario de América. Raíz lingüística: náhuatl (cacahuatl).
        \item Papaya - Fruto originario de América, con pulpa de color naranja intenso y semillas negras. Raíz lingüística: taíno (papáia).
        \item Guaraná - Planta originaria de la Amazonía, cuyas semillas se utilizan para hacer bebidas energéticas. Raíces lingüísticas: guaraní (waraná).
        \item Mate - Infusión de yerba mate, bebida tradicional en Argentina, Uruguay y Paraguay. Raíz lingüística: quechua (mati) y guaraní (mati).
        \item Alpaca - Animal similar a una llama, cuya lana se utiliza para hacer prendas de vestir. Raíz lingüística: aymara (allpaqa).
        \item Pisco - Bebida alcohólica destilada de uva, típica de Chile y Perú. Raíz lingüística: quechua (pisqu).
        \item Puma - Felino grande y fuerte, símbolo de poder en muchas culturas prehispánicas. Raíces lingüísticas: quechua (puma) y aimara (p'uma).
        \item Tamal - Preparación culinaria hecha a base de masa de maíz rellena de carne o vegetales, y cocida en hojas de maíz o plátano. Raíz lingüística: náhuatl (tamalli).
    \end{enumerate}

\subsection*{Raíz africana:}
    \begin{enumerate}
        \item Banano - Raíz lingüística: Wolof (lengua hablada en Senegal y Gambia).
        \item Chévere - Raíz lingüística: Mandinga (lengua hablada en África Occidental).
        \item Quilombo - Raíz lingüística: Kimbundu (lengua hablada en Angola).
        \item Zanahoria - Raíz lingüística: Árabe (a través del español medieval, pero con influencias africanas).
        \item Zombi - Raíz lingüística: Bantú.
    \end{enumerate}
\end{document}